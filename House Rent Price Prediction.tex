\documentclass[conference]{IEEEtran}[10]
\IEEEoverridecommandlockouts
\usepackage{cite}
\usepackage{amsmath,amssymb,amsfonts}
\usepackage{graphicx}
\usepackage{textcomp}
\usepackage{xcolor}
\usepackage{float}
\usepackage{anyfontsize}
\usepackage {hyperref}
\usepackage{authblk}
\begin{document}
\title{House Rent Prediction using polynomial and linear regression}
\author[1]{ Amit Roy
}
\affil[1]{ID: 2017-3-60-021
}

\author[2]{ Sirajum Maria Muna
}
\affil[2]{ID: 2017-3-60-020
}

\author[3]{ Shekhor Chandra Saha
}
\affil[3]{ID: 2017-3-60-025
}

\author[4]{ Saniat Injam
}
\affil[4]{ID: 2017-3-60-093
}






\maketitle
\IEEEpubidadjcol
\section{Introduction}
\subsection{1.1 Objectives}
With the mass growth of population around the world, it has become a challenge to provide accommodation for the people. As well as the growth of population the housing prices are also becoming high. For this reason, people opting for house renting as, owning a house seem like a dream like to some people. In this project our aim is to find house rent for a certain area in different cities. We are using a combination of linear regression and polynomial regression to predict house rent prices. 
\subsection{1.2 Motivation}
As we live in a third world country, also a over populated one, finding a good place to live a life peacefully is not easy. Along with ourselves, we want to secure a good future for our next generations. But money is always a problem. So, in this project we are motivated by this idea and tried to do a better approach, so that, it can help us by predicting the rental price to make us learn, on which area has better facilities in our suited budget. If anyone wants to further invest on real estates, this approach can help them. Also, it can be the parameter to signify one area’s demand.

\end{document}